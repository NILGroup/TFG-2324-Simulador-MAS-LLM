\chapter*{Resumen}

\section*{\tituloPortadaVal}

En la actualidad, están surgiendo multitud de aplicaciones que usan tecnologías apoyadas por grandes modelos de lenguaje que facilitan la vida a profesionales de todos los sectores. Estos modelos tienen un carácter general, por lo que permiten a los usuarios especializarse y particularizar en ciertos casos, para generar sistemas específicos para un sector.

Navegando por Internet, nos encontramos con el estudio de \ga  \citep{park2023generative}, el cual llamó nuestra atención desde que lo leímos por primera vez.

Este estudio desarrolló la ejecución de experimentos y simulaciones multi-agente de Inteligencia Artificial, y su objetivo final era estudiar la emergencia de fenómenos y relaciones sociales entre los agentes. En él se concluyó que efectivamente los agentes eran capaces de retener recuerdos, interactuar con otros agentes, generar nuevas memorias y planear sus próximas acciones.

Al ver esto, decidimos adentrarnos en este artículo, y descubrimos que había fascinado a toda la comunidad investigadora de Inteligencia Artificial, ya que aportaba un valor novedoso y demostraba que estas simulaciones eran coherentes. Por lo que varios de estos grupos de investigadores que se adentraron a realizar extensiones al programa original, enfocando las simulaciones para que hagan otro tipo de acciones (programar videojuegos autónomamente, asignando roles a cada agente, por ejemplo).

Viendo estas extensiones al código original, nos preguntamos si podríamos hacer algo similar nosotros, y hacerlo como TFG. Tras adentrarnos en el código e investigar el funcionamiento completo de la aplicación, tuvimos claro el caso de uso que quisimos enfocar: la democratización de esta herramienta y permitir que perfiles como psicólogos puedan usarla fácilmente.

Lo vimos claro porque a los perfiles a los que más les puede interesar una aplicación como esta es a psicólogos y profesionales que estudian las relaciones sociales. Sin embargo, tal y como estaba inicialmente diseñado el proyecto, era bastante complicado que un perfil no tecnológico, supiese cómo usar esta aplicación. Por lo que teniendo clara nuestra motivación y objetivo principal, nos pusimos manos a la obra.\\[0.1em]

El código desarrollado durante la realización de este trabajo se encuentra disponi- ble en https://github.com/NILGroup/TFG-2324-Simulador-MAS-LLM


\section*{Palabras clave}
   
\noindent Modelos de lenguaje grandes, comportamientos emergentes, interacción, interfaz, democratización de la tecnología, agentes inteligentes

   


