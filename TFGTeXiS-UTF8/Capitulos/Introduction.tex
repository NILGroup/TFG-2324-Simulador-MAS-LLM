\chapter*{Introduction}
\label{cap:introduction}
\addcontentsline{toc}{chapter}{Introduction}

\chapterquote{Creativity is just connecting things}{Steve Jobs}

Social relations have been a subject of study in multiple occasions throughout the years. Human behavior is certainly complex and, in many cases, unpredictable. This is what makes us humans unique and inimitable... or not?

In the past years, and specially last year, we have seen a dazzling development of Artificial Intelligences. One of the most interesting use cases for these technologies are the general purpose Large Language Models, which allow us to interact through natural language directly with the AI.

When developing these Language Models, if there is a big enough example dataset, the program will be able to generate completely unique and original situations, imitating already known examples, but without reusing them word by word, more like reinterpreting the knowledge.

In this moment human behavior comes into play. If we have some very large models, which are able to follow input schemas and create unique situations, would we be able to simulate plausible human relations by using these technologies?

This experiment was performed in the \textit{Generative Agents} study, in which we have focused to create our Final Degree Thesis. In this experiment, they succeed on creating independent agents that can interact among themselves, they have memory of past events and are able to develop social relations between one another.

\section{Motivation}
In response to the expectations of the capabilities of Large Language Models, a lot of research emerges that explores the limits of these technologies.

Among these works is \ga, in which an architecture based on the use of Large Language Models is proposed, with which to create agents capable of simulating a human personality. Maintaining coherence between their personality, previous interactions and motivations.

With an architecture like the one mentioned, we have what is necessary to create an environment in which several agents intervene and interact with each other. An implementation of this idea is what we have in the repository \textit{generative\_agents} created by the authors of the paper mentioned above.

Having a technology capable of simulating human behavior in social contexts creates the need for us to experiment with these capabilities in different situations. This need is reflected in works such as \textit{Communicative Agents for Software Development} which tests the capabilities of \ga, but this time measuring the ability of a group of agents to generate software. In this study, the same agents successfully associate, build teams and generate functional videogames with real code.

The potential of this technology and the interest of people in defining its limits, applications or simply experimenting is thus seen. However, all these motivations are very limited when it comes to putting this work into operation in new scenarios.

That is why we propose in this Final Degree Thesis to facilitate access and interaction with the technology that they provide in \ga, in addition to offering new functionalities with which to explore and interact with the simulation in real time.

\section{Objectives}

\section{Document structure}







