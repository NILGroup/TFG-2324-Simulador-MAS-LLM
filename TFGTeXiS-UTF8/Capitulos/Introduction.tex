\chapter*{Introduction}
\label{cap:introduction}
\addcontentsline{toc}{chapter}{Introduction}

\chapterquote{Creativity is just connecting things}{Steve Jobs}

Social relations have been a subject of study in multiple occasions throughout the years. Human behavior is certainly complex and, in many cases, unpredictable. This is what makes us humans unique and inimitable... or not?


En los últimos años, y especialmente en el último, hemos visto un desarrollo fulgurante de las Inteligencias Artificiales. Uno de los casos de uso más interesante de estas tecnologías son los Modelos Grandes del Lenguaje (Large Language Models), de propósito general, los cuales nos permiten interactuar mediante el lenguaje natural directamente con la IA.

Al desarrollar estos Modelos del Lenguaje, si se tiene una serie de ejemplos lo suficientemente grande, el programa podrá generar situaciones completamente únicas y originales, imitando ejemplos ya conocidos pero sin reutilizarlos de manera literal.

En este momento entra en juego el comportamiento humano. Si tenemos unos Modelos muy grandes, capaces de obedecer a esquemas y crear situaciones únicas, ¿podríamos simular relaciones sociales humanas verosímiles usando estas tecnologías?

Este experimento fue llevado a cabo en el estudio de \textit{Generative Agents}, en el cual nos hemos basado a la hora de realizar nuestro Trabajo de Fin de Grado. En este experimento logran crear agentes independientes los cuales interaccionan entre ellos, tienen recuerdos de eventos ocurridos y son capaces de desarrollar relaciones sociales entre sí.

Introduction to the subject area. This chapter contains the translation of Chapter \ref{cap:introduccion}.








