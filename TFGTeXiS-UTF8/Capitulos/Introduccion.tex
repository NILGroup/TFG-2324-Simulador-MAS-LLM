\chapter{Introducción}
\label{cap:introduccion}

\chapterquote{La creatividad es únicamente unir conceptos}{Steve Jobs}

Según la normativa  para Trabajos de Fin de Grado\footnote{\url{https://informatica.ucm.es/file/normativatfg-2021-2022?ver} (ver versión actualizada para cada curso académico)}


Las relaciones sociales han sido tema de estudio en múltiples ocasiones a lo largo de los años. El comportamiento de los seres humanos es ciertamente complejo y, en muchos casos, impredecible. Esto es lo que nos hace únicos e inimitables... ¿o no?

En los últimos años, y especialmente en el último, hemos visto un desarrollo fulgurante de las Inteligencias Artificiales. Uno de los casos de uso más interesante de estas tecnologías son los Modelos Grandes del Lenguaje (Large Language Models), de propósito general, los cuales nos permiten interactuar mediante el lenguaje natural directamente con la IA.

Al desarrollar estos Modelos del Lenguaje, si se tiene una serie de ejemplos lo suficientemente grande, el programa podrá generar situaciones completamente únicas y originales, imitando ejemplos ya conocidos pero sin reutilizarlos de manera literal.

En este momento entra en juego el comportamiento humano. Si tenemos unos Modelos muy grandes, capaces de obedecer a esquemas y crear situaciones únicas, ¿podríamos simular relaciones sociales humanas verosímiles usando estas tecnologías?

Este experimento fue llevado a cabo en el estudio de \textit{Generative Agents}, en el cual nos hemos basado a la hora de realizar nuestro Trabajo de Fin de Grado. En este experimento logran crear agentes independientes los cuales interaccionan entre ellos, tienen recuerdos de eventos ocurridos y son capaces de desarrollar relaciones sociales entre sí.

\section{Motivación}
Introducción al tema del TFG.


\section{Objetivos}
Descripción de los objetivos del trabajo.


\section{Plan de trabajo}
Aquí se describe el plan de trabajo a seguir para la consecución de los objetivos descritos en el apartado anterior.



\section{Explicaciones adicionales sobre el uso de esta plantilla}
Si quieres cambiar el \textbf{estilo del título} de los capítulos del documento, edita el fichero \verb|TeXiS\TeXiS_pream.tex| y comenta la línea \verb|\usepackage[Lenny]{fncychap}| para dejar el estilo básico de \LaTeX.

Si no te gusta que no haya \textbf{espacios entre párrafos} y quieres dejar un pequeño espacio en blanco, no metas saltos de línea (\verb|\\|) al final de los párrafos. En su lugar, busca el comando  \verb|\setlength{\parskip}{0.2ex}| en \verb|TeXiS\TeXiS_pream.tex| y aumenta el valor de $0.2ex$ a, por ejemplo, $1ex$.

TFGTeXiS se ha elaborado a partir de la plantilla de TeXiS\footnote{\url{http://gaia.fdi.ucm.es/research/texis/}}, creada por Marco Antonio y Pedro Pablo Gómez Martín para escribir su tesis doctoral. Para explicaciones más extensas y detalladas sobre cómo usar esta plantilla, recomendamos la lectura del documento \texttt{TeXiS-Manual-1.0.pdf} que acompaña a esta plantilla.

El siguiente texto se genera con el comando \verb|\lipsum[2-20]| que viene a continuación en el fichero .tex. El único propósito es mostrar el aspecto de las páginas usando esta plantilla. Quita este comando y, si quieres, comenta o elimina el paquete \textit{lipsum} al final de \verb|TeXiS\TeXiS_pream.tex|

\subsection{Texto de prueba}


\lipsum[2-20]