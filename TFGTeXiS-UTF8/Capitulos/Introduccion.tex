\chapter{Introducción}
\label{cap:introduccion}

\chapterquote{La creatividad es únicamente unir conceptos}{Steve Jobs}

Las relaciones sociales han sido tema de estudio en múltiples ocasiones a lo largo del tiempo. El comportamiento de los seres humanos es ciertamente complejo y, en muchos casos, impredecible. Por ello, surge la necesidad de tratar de comprender la base de estas relaciones e intentar predecir los comportamientos mediante las simulaciones.

En los últimos años, y especialmente en el último, hemos visto un desarrollo fulgurante de la Inteligencia Artificial generativa. Uno de los casos de uso más interesante de estas tecnologías son los Modelos Grandes de Lenguaje (Large Language Models), de propósito general, los cuales nos permiten interactuar mediante el lenguaje natural directamente con estos sistemas de IA generativa.

Al desarrollar estos Modelos del Lenguaje, si se tiene una serie de ejemplos lo suficientemente grande, el programa podrá generar situaciones completamente únicas y originales, imitando ejemplos ya conocidos pero sin reutilizarlos de manera literal.

En este momento entra en juego el comportamiento humano. Si tenemos unos modelos muy grandes, capaces de obedecer a esquemas de entrada y crear situaciones únicas, ¿podríamos simular relaciones sociales humanas verosímiles usando estas tecnologías?

Este experimento fue llevado a cabo en el estudio de \cite{park2023generative}, en el cual nos hemos basado a la hora de realizar nuestro Trabajo de Fin de Grado. En este experimento logran crear agentes independientes los cuales interaccionan entre ellos, tienen recuerdos de eventos ocurridos y son capaces de desarrollar relaciones sociales entre sí.

\section{Motivación}
La motivación inicial de este trabajo era poder crear historias a partir de simulaciones de agentes de Inteligencia Artificial. Sin embargo, al encontrar el estudio de \ga \citep{park2023generative}, decidimos cambiar el objetivo y centrarnos en cómo gente de perfiles no tecnológicos, especialmente psicólogos, pueden estudiar las relaciones sociales entre humanos, partiendo de una simulación que ya funcionaba y lograba que los personajes tuviesen recuerdos e interaccionasen de manera verosímil.

Por tanto, a partir de ese momento, nuestro principal objetivo se convirtió en crear una aplicación en la cual psicólogos u otro tipo de usuarios puedan estudiar relaciones sociales indicando diferentes situaciones y facilitando los cambios entre simulaciones para estudiar las diferencias de comportamiento.

Con un sistema como el citado, disponemos de lo necesario para crear un entorno en el que intervengan varios agentes e interactúen entre sí. Una implementación de esta idea es la que tenemos en el repositorio {\textit{generative\_agents}}\footnote{\url{https://github.com/joonspk-research/generative_agents}} creado por los autores del paper mencionado anteriormente.

Disponer de una tecnología capaz de simular el comportamiento humano en contextos sociales nos crea la necesidad de experimentar con estas capacidades en distintas situaciones. Esta necesidad se ve reflejada en otro estudio, el cual pone a prueba y expande el mismo trabajo original, de \cite{park2023generative}, pero en esta ocasión midiendo la capacidad de un grupo de agentes para generar software \citep{qian2023communicative}. En este estudio se comprueba que estos mismos agentes son capaces de asociarse, formar equipos y generar videojuegos funcionales con código real.

Se ve así el potencial de esta tecnología y el interés de la gente por definir sus límites, aplicaciones o simplemente experimentar. Sin embargo todas estas motivaciones se ven muy coartadas a la hora de poner en funcionamiento este trabajo en nuevos escenarios debido a la complejidad de la ejecución de la aplicación, la cual es difícil de usar para perfiles ajenos al mundo de la programación, y muy difícilmente extensible debido a la situación inicial del código (personajes y mapas preseleccionados y no intercambiables, rigidez a la hora de ejecutar simulaciones, falta de modificación del contexto de la aplicación...).

Es por ello que proponemos en este Trabajo de Fin de Grado facilitar el acceso e interacción con la tecnología que proporcionan en \ga, además de ofrecer nuevas funcionalidades con las que explorar e interactuar con la simulación en tiempo real.

\section{Objetivos}

Para lograr el objetivo principal del TFG mencionado anteriormente, que es facilitar el acceso e interacción y extender funcionalidades del estudio de  \ga, hemos decidido estructurarlo en los siguientes puntos:

\begin{itemize}
\item Creación de simulaciones configurando agentes y/o el escenario. Ya que actualmente solamente existe un mapa predeterminado y una serie de agentes predefinidos. La finalidad es permitir una mayor interactividad con el entorno, añadiendo mapas y pudiendo personalizar a los agentes.

\item Permitir la interacción con el estado de los agentes durante la simulación. En cualquier momento, el usuario podrá interactuar con los agentes, chateando en tiempo real con ellos o 'susurrándoles' las próximas acciones que deben tomar.

\item Visión de la simulación, a través de un personaje o de una forma general y sintetizada. También permitir a los usuarios, una vez finalizada una simulación, poder hacer un resumen de toda la simulación como conjunto o hacerlo desde el punto de vista de uno de los agentes.

\item Gestión de las simulaciones, permitiendo el guardado de estas y la creación en base a anteriores. Esto permitirá poder ver simulaciones repetidas o extenderlas.

\item Integración de todo lo anterior en una interfaz web nueva y creada completamente desde cero, comunicándose con el back en tiempo real y sin necesidad de usar una terminal para comunicarnos.

\item Integración con el modelo de lenguaje y la API de OpenAI actualizada, ya que las llamadas originales usaban una API anterior que no funcionaba correctamente.

\end{itemize}


\section{Plan de trabajo}
Con el fin de cumplir los objetivos planteados en la sección anterior, se ha fijado una planificación dividida en las siguientes etapas:

\begin{itemize}
	
	\item \textbf{Etapa de investigación y definición de objetivos}. En esta primera parte del desarrollo, nos centraremos en investigar artículos científicos y libros que traten sobre temas similares a la interacción de agentes de IA, así como definiendo los objetivos básicos, en qué sentido vamos a extender el trabajo descrito en el artículo original de \ga y comenzar a redactar esbozos iniciales de la introducción y estado de la cuestión de la memoria.
	
	\item Tras concluir con las investigaciones iniciales y la adaptación al tema, comenzaremos la fase del \textbf{desarrollo del proyecto}, realizando así los diseños de las distintas interfaces, el desarrollo de las mismas y la producción de los capítulos principales de la memoria.
	
	\item Finalmente, llegaremos a la fase de \textbf{finalización y revisión del desarrollo}, en la cual finalizaremos todos los detalles y objetivos adicionales que queramos incluir en el proyecto, una vez finalizados los objetivos propuestos. En esta etapa también añadiremos el resumen, resultados, comclusiones y trabajo futuro, así como también revisaremos el resultado final de la memoria y el trabajo en su totalidad.
\end{itemize}



\section{Estructura del documento}
El presenta documento está formado por siete capítulos, incluyendo este capítulo introductorio como el primero de ellos con el fin de conocer la motivación que lleva al desarrollo del proyecto y los objetivos de este. Además, hay un último capítulo en el qe cada uno de los estudiantes explicamos el trabajo que hemos realizado individualmente. A continuación, se presentan el resto de capítulos que componen la memoria: 

\begin{itemize}
	
	\item \textbf{Capítulo 2}: En el Estado de la Cuestión se tratan temas que aportan un contexto importante al presente trabajo. Por una parte se abordan temas técnicos, como los agentes inteligentes, las simulaciones y los modelos de lenguaje, por otro lado, la unión entre la tecnología y el comportamiento humano, con el procesamiento del lenguaje natural y la computación centrada en el humano, y finalmente se tratan temas relacionados puramente con las relaciones sociales y la democratización de sistemas informáticos.
	
	\item \textbf{Capítulo 3}: En este capítulo se abordan las diferencias entre las funcionalidades que ya existían en el sistema que empleamos, las que existían pero las ampliamos, y las nuevas que creamos durante el presente trabajo. En este capítulo solo se aborda el trabajo desde un punto de vista de funcionalidades, a alto nivel y no el estado tecnológico a bajo nivel.
	
	\item \textbf{Capítulo 4}: En el Planteamiento de la Solución se tratan varios puntos para aportar contexto y recalcar hitos que han ido ocurriendo durante el desarrollo. Por un lado, se aporta desde un punto de vista tecnológico y de bajo nivel el estado del sistema inicialmente. Tras esto, se tratan todos los ajustes, cambios y reconsideraciones que acontecieron mientras se desarrollaba el propio trabajo, así como problemas encontrados y cómo se han solucionado.
	
	\item \textbf{Capítulo 5}: En el cual se tratan todas las extensiones que se han realizado al sistema original, a bajo nivel y explicando los cambios, su importancia y el motivo.
	
	\item \textbf{Capítulo 6}: Desde el punto de vista de la interfaz, todos los cambios, nuevas vistas y páginas que han sido creadas para que la aplicación pueda ser completamente utilizada desde la web, ya que originalmente había que ejecutar comandos en la terminal para realizar ciertas acciones.
	
	\item \textbf{Capítulo 7}: En el capítulo de Conclusiones y Trabajo Futuro se añaden posibles extensiones al sistema u objetivos que habríamos implementado si hubiéramos tenido más tiempo, a tener en cuenta para el futuro.
	
\end{itemize}
