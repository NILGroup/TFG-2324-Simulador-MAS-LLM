\chapter{Introducción}
\label{cap:introduccion}

\chapterquote{La creatividad es únicamente unir conceptos}{Steve Jobs}

Las relaciones sociales han sido tema de estudio en múltiples ocasiones a lo largo del tiempo. El comportamiento de los seres humanos es ciertamente complejo y, en muchos casos, impredecible. Esto es lo que nos hace únicos e inimitables... ¿o no?

En los últimos años, y especialmente en el último, hemos visto un desarrollo fulgurante de las Inteligencias Artificiales. Uno de los casos de uso más interesante de estas tecnologías son los Modelos Grandes de Lenguaje (Large Language Models), de propósito general, los cuales nos permiten interactuar mediante el lenguaje natural directamente con la IA.

Al desarrollar estos Modelos del Lenguaje, si se tiene una serie de ejemplos lo suficientemente grande, el programa podrá generar situaciones completamente únicas y originales, imitando ejemplos ya conocidos pero sin reutilizarlos de manera literal.

En este momento entra en juego el comportamiento humano. Si tenemos unos modelos muy grandes, capaces de obedecer a esquemas de entrada y crear situaciones únicas, ¿podríamos simular relaciones sociales humanas verosímiles usando estas tecnologías?

Este experimento fue llevado a cabo en el estudio de \ga\footnote{\url{https://arxiv.org/abs/2304.03442}}, en el cual nos hemos basado a la hora de realizar nuestro Trabajo de Fin de Grado. En este experimento logran crear agentes independientes los cuales interaccionan entre ellos, tienen recuerdos de eventos ocurridos y son capaces de desarrollar relaciones sociales entre sí.

\section{Motivación}
En respuesta a las expectativas de las capacidades de los Modelos Grandes de Lenguaje surgen una gran cantidad de investigaciones que exploran los límites de estas tecnologías.

Entre estos trabajos se encuentra \ga, en el que se propone una arquitectura basada en el uso de Modelos Grandes de Lenguaje, con la que crear agentes capaces de simular una personalidad humana. Manteniendo coherencia entre su personalidad, interacciones previas y motivaciones.

Con una arquitectura como la citada disponemos de lo necesario para crear un entorno en el que intervengan varios agentes e interactúen entre sí. Una implementación de esta idea es la que tenemos en el repositorio \href{https://github.com/joonspk-research/generative_agents}{\textit{generative\_agents}} creado por los autores del paper mencionado anteriormente.

Disponer de una tecnología capaz de simular el comportamiento humano en contextos sociales nos crea la necesidad de experimentar con estas capacidades en distintas situaciones. Esta necesidad se vé reflejada en trabajos como \textit{Communicative Agents for Software Development}\footnote{\url{https://arxiv.org/abs/2307.07924}}, que pone a prueba las capacidades de \ga, pero en esta ocasión midiendo la capacidad de un grupo de agentes para generar software. En este estudio se comprueba que estos mismos agentes son capaces de asociarse, formar equipos y generar videojuegos funcionales con código real.

Se vé así el potencial de esta tecnología y el interés de la gente por definir sus límites, aplicaciones o simplemente experimentar.
Sin embargo todas estas motivaciones se ven muy coartadas a la hora de poner en funcionamiento este trabajo en nuevos escenarios.

Es por ello que proponemos en este Trabajo de Fin de Grado facilitar el acceso e interacción con la tecnología que proporcionan en \ga, además de ofrecer nuevas funcionalidades con las que explorar e interactuar con la simulación en tiempo real.

\section{Objetivos}
Para lograr el objetivo principal del TFG mencionado anteriormente, que es facilitar el acceso e interacción y extender funcionalidades del estudio de  \ga, hemos decidido seccionarlo en los siguientes puntos:
\begin{itemize}
\item Creación de simulaciones configurando agentes y/o el escenario.
\item Permitir la interacción con el estados de los agentes y escenario durante la simulación.
\item Visión de la simulación, a través de un personaje o de una forma general y sintetizada.
\item Gestion de las simulaciones, permitiendo el guardado de estas y la creación en base a anteriores.
\item Integración de todo lo anterior en una interfaz web.
\item Creación de sistema que permita usar modelos de lenguaje que se ejecuten de forma local, o APIs en su defecto.
\end{itemize}


\section{Plan de trabajo}
Aquí se describe el plan de trabajo a seguir para la consecución de los objetivos descritos en el apartado anterior.



\section{Estructura del documento}
El presenta documento está formado por  \textbf{A RELLENAR}  capítulos, incluyendo este capítulo introductorio como el primero de ellos con el fin de conocer la motivación que lleva al desarrollo del proyecto y los objetivos de este. A continuación, se presentan el resto de capítulos que componen la memoria: 

Si quieres cambiar el \textbf{estilo del título} de los capítulos del documento, edita el fichero \verb|TeXiS\TeXiS_pream.tex| y comenta la línea \verb|\usepackage[Lenny]{fncychap}| para dejar el estilo básico de \LaTeX.

Si no te gusta que no haya \textbf{espacios entre párrafos} y quieres dejar un pequeño espacio en blanco, no metas saltos de línea (\verb|\\|) al final de los párrafos. En su lugar, busca el comando  \verb|\setlength{\parskip}{0.2ex}| en \verb|TeXiS\TeXiS_pream.tex| y aumenta el valor de $0.2ex$ a, por ejemplo, $1ex$.

TFGTeXiS se ha elaborado a partir de la plantilla de TeXiS\footnote{\url{http://gaia.fdi.ucm.es/research/texis/}}, creada por Marco Antonio y Pedro Pablo Gómez Martín para escribir su tesis doctoral. Para explicaciones más extensas y detalladas sobre cómo usar esta plantilla, recomendamos la lectura del documento \texttt{TeXiS-Manual-1.0.pdf} que acompaña a esta plantilla.

El siguiente texto se genera con el comando \verb|\lipsum[2-20]| que viene a continuación en el fichero .tex. El único propósito es mostrar el aspecto de las páginas usando esta plantilla. Quita este comando y, si quieres, comenta o elimina el paquete \textit{lipsum} al final de \verb|TeXiS\TeXiS_pream.tex|

\subsection{Texto de prueba}


\lipsum[2-20]