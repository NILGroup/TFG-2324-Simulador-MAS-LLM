\chapter{Conclusiones y Trabajo Futuro}
\label{cap:conclusiones}

En el presente trabajo se ha desarrollado un sistema que permite el acceso a la simulación multiagente para el estudio de las relaciones sociales e investigación de la emergencia de fenómenos sociales. Esto permitirá que sea una herramienta muy útil para que cualquier tipo de usuario sin conocimiento de informática, como psicólogos (especialmente interesados en las simulaciones sociales) puedan analizar el comportamiento humano simulado, y se puedan predecir ciertas situaciones.

En el presente capítulo se describen tres subapartados. El primero de ellos es una especie de discusión en la que, tras exponer el sistema original en el capítulo del Estado de la Cuestión y tras ver las implementaciones y trabajo que hemos realizado a lo largo del resto de capítulos, pretendemos mostrar el impacto que los cambios realizados han tenido en el resultado final del sistema, mostrando así las diferencias en la aplicación al principio respecto a ahora. En el segundo subapartado se tratan las conclusiones tras analizar el trabajo una vez finalizado. Finalmente, en el último subapartado se profundizan en algunas de las posibles futuras extensiones que se habrían incluido en este trabajo si se hubiese dispuesto del tiempo necesario, ya que sería interesante verlas implementadas en el futuro.

\section{Diferencias entre el Sistema Original y el Actual}

Teniendo en cuenta las funcionalidades que existían originalmente en el sistema, discutidas en profundidad en la sección \ref{genAgents}, y conociendo la serie de desarrollos que se han realizado para modificar esta aplicación discutidos en los capítulos anteriores, dividimos estas modificaciones en dos apartados: las funcionalidades ampliadas y las nuevas funcionalidades.

\subsection{Funcionalidades ampliadas}
Este primer apartado aborda las funcionalidades que ya se encontraban originalmente desarrolladas. Para estas, existen algunas nuevas características que se han implantado durante el desarrollo de este trabajo que se basan en ellas para funcionar, las cuales son las mencionadas a continuación.

\subsubsection{Susurros a los personajes mediante la interfaz}
Como se mencionó en el apartado \ref{genAgents}, la funcionalidad de susurrar a los personajes ya estaba implementada en el sistema original. Sin embargo, para poder hacer esto era necesario acudir a la terminal y desde ahí seleccionar el personaje y la información. 

Tal y como se ha extendido, se podrá acceder a uno de los personajes clicando en su persona en la parte inferior de la simulación y una vez hecho esto, habrá un botón en el que el usuario se podrá comunicar directamente y susurrar la acción o la personalidad que quiera que interprete el personaje.

Este flujo de acción se ha visto en el capítulo 5 de la interfaz, más específicamente en el apartado  \ref{botonSusurro}, donde se puede ver el botón de susurro y el modal resultante al clicar este botón.

\subsubsection{Nuevas vistas}
Además de las vistas existentes, se han creado nuevas páginas, de modo que la aplicación se pueda usar exclusivamente desde la interfaz gráfica, sin necesidad de acceder a la terminal (esto ayuda a usar la aplicación a las personas no técnicas) y uniéndolas todas en diferentes vistas.

Por un lado, se han creado vistas como la landing, una página para crear simuaciones desde la web, y otras páginas para visualizar y continuar simulaciones para poder acceder directamente desde ahí a ellas, simplemente clicando. Por el otro, también se han enriquecido las vistas ya existentes añadiendo botones como los de susurrar o chatear con los agentes durante la siulación.

Además, se ha añadido una barra de navegación mediante la cual se podrá navegar por la aplicación y acceder a todas las vistas necesarias para ejecutar y visualizar simulaciones.

\subsection{Nuevas funcionalidades}
Finalmente, existen funcionalidades que no existían en la aplicación original y hemos añadido, aportando así valor al sistema ya existende y reiterando su importancia para el uso en investigación psicológica.

\subsubsection{Interacción total mediante la interfaz}
Además de lo visto anteriormente, en cuanto al botón de susurrar a los personajes, y siguiendo con el objetivo principal de acercar esta herramienta a perfiles menos tecnológicos, una adaptación que se ha realizado es que todas las interacciones con la herramienta se pueden desarrollar desde el frontend de la aplicación.

Antes era necesario crear una instancia del backend, mediante la cual conocer los archivos que componían el sistema e interactuar con el programa mediante la terminal. Ahora, los usuarios serán agnósticos de la implementación y cuando se ejecute una simulación, se creará e interactuará con la terminal mediante el frontend. Esta funcionalidad está explicada a bajo nivel en el capítulo \ref{cap:extensiones}.

\subsubsection{Función de chat en tiempo real con los agentes}
Similar a la funcionalidad de susurro, hemos implementado una nueva característica en la que los usuarios podrán, en medio de una simulación, chatear con cualquiera de los personajes para preguntarles su opinión sobre ciertos temas, para ver cómo va avanzando a medida que avanza la simulación.

En este caso, se diferencia del susurro ya que la interacción en el chat no influirá en el comportamiento del agente en la simulación.

Al igual que ocurría con el botón de susurro, el flujo de interacción con el botón y la ventana emergente del chat se puede ver en el apartado \ref{botonSusurro}.

\subsubsection{Guardado y gestión de nuevas simulaciones}

En cuanto al guardado de simulaciones, tal y como estaba organizado inicialmente, mediante el uso de los comandos se podía salir de la ejecución de la simulación y esta se guardaba en una carpeta dentro del propio sistema del usuario. El problema a la hora de volver a estas simulaciones, tanto para visualizarlas como para continuarlas, era que se guardaban todas en la misma carpeta, y era difícil de recuperarlas, ya que tenías que añadir el nombre exacto de la simulación mediante el uso de la consola.

Ahora, las simulaciones se seguirán guardando en el mismo sitio y reteniendo la misma información, pero para recuperarlas y poder ejecutarlas o verlas, estarán disponibles en una sola vista todas las que el usuario tenga guardadas, y con un simple clic podrá acceder a ellas para realizar la acción que desee. Ya sea continuar con la simulación o visualizarla. Además, en el apartado de visualizar simulación, también habrá un link que redirigirá a los usuarios a la simulación de la que parte esa repetición, para que así puedan modificarla a su gusto si quisiesen experimentar con ella.

Además, se añade la funcionalidad de la gestión de las nuevas simulaciones, pudiendo crearlas y editarlas fácilmente, así como visualizarlas, ejecutarlas o modificarlas puramente desde la interfaz gráfica.

\subsubsection{Resúmenes de simulaciones}

Para aportar valor y contexto a las simulaciones, se ha añadido la función de la generación automática de resúmenes. Una vez el usuario realiza una simulación y la desea guardar para poder visualizarla en el futuro, dentro del proceso de guardado de la simulación y compresión de los datos, también se generará un resumen automáticamente, llamando al modelo de lenguaje para ello, en el que se indique a alto nivel lo ocurrido durante la simulación y los momento en los que ocurren las situaciones más interesantes a estudiar (momentos de interacción entre agentes, en los que el usuario ha interactuado con ellos, en los que cambia algo importante en la simulación, etc.) para que el investigador pueda visualizar los momentos más interesantes o modificar la simulación desde ese punto.

Estos resúmenes estarán disponibles una vez se finalice y se guarde la simulación, a modo de indicativo sobre lo que ha ocurrido durante esta a alto nivel. Si tras leer el resumen el usuario quiere modificar la simulación, puede tomar uno de los tiempos de referencia y editar la simulación desde ese punto, para así estudiar las diferencias dependiendo de qué cambie a partir de ciertas modificaciones en determinados puntos clave de la simulación.\\

\section{Conclusiones}

La intención inicial del presente Trabajo de Fin de Grado era la democratización del uso de un sistema de simulaciones multi agentes, el cual ya existía previamente, para que perfiles como psicólogos pudiesen estudiar los fenómenos sociales y experimentar con las relaciones entre diferentes agentes de Inteligencia Artificial.

Para ello, uno de los requisitos principales era cambiar la arquitectura de la aplicación, de manera que, en lugar de interactuar con el sistema mediante una terminal, los usuarios interactuasen mediante la interfaz. La interfaz es uno de los puntos principales de la aplicación final, ya que es el punto en el que los usuarios realizan todas las acciones ahora, desde crear simulaciones hasta visualizarlas o interactuar con los agentes en tiempo real.

Además del desarrollo de la propia interfaz, fue necesario un exhaustivo trabajo en la manera en la que se ejecutaba la aplicación. Haciendo así una reestructuración total del diseño y parte de la arquitectura del sistema. Esta reestructuración viene dada ya que ahora será necesario capturar los comandos mediante la interfaz, por lo que debemos tener un sistema corriendo el cual escuche constantemente a si el usuario ejecuta un comando y, en caso positivo, comunicarlo al backend de reverie y que la aplicación pueda seguir funcionando mientras este calcula la respuesta. Antes, esto era muy diferente ya que se interactuaba directamente con la interfaz y no existía esta problemática.

Además de centrarnos en la democratización del uso del sistema, que era el epicentro del trabajo, también concluímos que podrían ser muy positivas ciertas extensiones sobre el propio trabajo preexistente. Las características que decidimos implementar finalmente fueron las de chat en tiempo real con el personaje y la generación automática de resúmenes tras guardar una simulación. Estas dos características no estaban inicialmente implementadas y, reaprovechando parte del código previamente existente, se pudieron implementar realizando una serie de llamadas al modelo del lenguaje. En la función de chat se usan los recuerdos particulares de un personaje para así poder interactuar con su conocimiento en ese momento, mientras que en los resúmenes automáticos, se usan las memorias de todos los personajes y el contexto de lo que ha ido ocurriendo para generar una historia al final de cada simulación.

Finalmente, se llegó a un resultado final del sistema en el que logramos cumplir todos los objetivos marcados (una vez se reconsideraron los mismos), habiendo así democratizado el uso de esta nueva herramienta para la investigación de los fenómenos sociales con agentes de Inteligencia Artificial. Cabe recalcar también que, a pesar de que el desarrollo está realizado, los costes de operar esta aplicación a día de hoy con la API de OpenAI como la que estamos usando son sumamente elevados. Esto es un problema para todos los sistemas multi agente en general. Si en unos años este coste se reduce o salen modelos lo suficientemente pequeños y potentes para ejecutar este código de manera verosímil, será una gran herramienta sumamente potente, pero las limitaciones tecnológicas a día de hoy no nos permiten demostrar todas las capacidades de la aplicación.

\section{Trabajo futuro}

A pesar de que cumplimos los objetivos y requisitos epecificados al inicio del desarrollo, también es cierto que a medida que se progresaba con el trabajo, iban surgiendo ciertas ideas que habrían tenido un impacto positivo en el proyecto si hubiese habido suficiente tiempo para implementar todas.

La decisión que tomamo a mitad de desarrollo fue la de priorizar ciertos objetivos y cumplir los que se habían propuesto para el marco de este proyecto. Una vez realizada esta priorización, hubo algunos de los objetivos que se quedaron fuera del desarrollo por falta de tiempo, a pesar de lo interesantes que podrían llegar a ser.

A continuación, se muestra una lista con algunos de los objetivos que nos habría gustado implementar y que son buena idea para realizar trabajos futuros:

\begin{itemize}
	\item \textbf{Elección y cambio de mapa}: Dado el contexto necesario, podríamos hacer que los personajes estuviesen desconectados del mapa. Sin embargo, tal y como estaba inicialmente el sistema, cada personaje se encontraba estrechamente arraigado al mapa, ya que partes del propio mapa tenían los nombres de los personajes. Una vez que se ha conseguido separar a los personajes del mapa, habría sido buena idea crear algún otro mapa alternativo, o permitir a los propios usuarios que creen sus propios mapas. Esto no tuvo tan alta prioridad ya que, como comentamos, los propios usuarios pueden hacer que los agentes sean agnósticos del mapa en el que están y simulen cualquier otro espacio
	
	\item \textbf{Personalización de los personajes}: Como se comentó en la sección \ref{problemaPersonajes}, el gran problema de los personajes es que estaban directamente ligados al mapa, y el nombre de estos era sumamente importante para guardar las simulaciones y que funcionase correctamente. Sin embargo, al añadir nuevos mapas, también se podría cambiar esto, y permitir a los propios usuarios que elijan los nombres de los personajes, así como elegir o editar cómo se verán cada uno de ellos.
	
	\item \textbf{Mejora de la interfaz gráfica}:A pesar de que se ha realizado un diseño específico para las funcionalidades de la aplicación, se podría haber experimentado con otra gama de colores o recursos estáticos para estilar las diferentes páginas de la aplicación, y la investigación con esto es sin duda un punto a mejorar en el futuro.
	
	\item \textbf{Interacción directa con el entorno}: En la aplicación original se permitía modificar ciertas partes del entorno. Es decir, se podía coger un objeto y cambiar su estado (por ejemplo, decir que la cama de un personaje estaba en llamas y ver cómo reaccionaban), sin embargo, traducir esto a la interfaz era sumamente complicado, ya que habría que entrar en la implementación de Phaser, que se reconozca cada objeto con los clicks y luego realizar la consulta de la modificación del entorno con el backend. Es un objetivo ambicioso que sería muy interesante de ver y aportaría mucho valor a la aplicación.
	
	\item \textbf{Libre elección de LLM}: Tras investigar multitud de modelos de lenguaje y probar con ellos, llegamos a la conclusión de que a día de hoy, con las condiciones que tenemos, la opción menos mala para ejecutar este sistema es mediante el uso de la API de OpenAI. Esta no permite realizar muchas llamadas, pero que es una de las pocas que comprende cómo procesar la entrada que enviamos, y que funciona bien a día de hoy. Sería interesante en el futuro que los propios usuarios puedan introducir sus modelos de lenguaje que tengan instalados en sus ordenadores, o que puedan utilizar cualquier otra API que tengan a sus disposición.
	
	\item \textbf{Más resúmenes}: Además de los resúmenes que se realizan automáticamente al guardar una simulación, sería buena idea que se guardasen una mayor cantidad de resúmenes para ayudar a los usuarios a ubicar cada simulación. Podría ser una buena idea tener resúmenes que indiquen en qué timestamps ocurren los eventos más interesantes, o que se realice un resumen por cada uno de los personajes de la aplicación automáticamente.

	\item\textbf{Bifurcación de simulaciones en cualquier instante de tiempo}

Con el objetivo de brindar una herramienta que permita crear situaciones pseudo-sociales realistas con el fin de analizar los fenómenos de diversidad de situaciones, vimos un interés singular en permitir la bifurcación de las simulaciones ya hechas en cualquier momento del tiempo.

Con esto se logra la posibilidad de explorar escenarios alternativos desde puntos críticos en el tiempo que, unido a la síntesis ofrecida por los resúmenes, abre la puerta a realizar simulaciones que generen situaciones imprevistas, aunque de interés, y una vez resumidas sean detectadas y objeto de bifurcación para observar el comportamiento del sistema ante cambios provistos por el usuario.

La implementación de esta funcionalidad implica guardar la información necesaria del sistema en cada momento del tiempo. De forma que se pueda replicar cualquier momento del tiempo del sistema. 

La forma naíz de hacerlo es replicar la información existente en cada momento del tiempo. Este enfoque es imposible en la práctica debido al tamaño de memoria que ocuparía una sola simulación. Estimando un poco, en base a una simulación de 3 agentes y un total de 24 horas, es decir, 8640 steps y ocupando un total de 57 MB. Si asumimos que el crecimiento en espacio es lineal respecto al numero de steps, y lo es, esto implicaría un coste cuadrático de espacio en el tiempo simulado. Llevando la simulación de 24 horas de tiempo a aproximádamente 1653 MB de espacio. Lo cual no es viable, más aún habiendo mejores formas de plantear la solución.	

\end{itemize}

